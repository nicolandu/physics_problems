\documentclass[letterpaper]{article}

\usepackage{lmodern}% use Latin Modern font, which is vector
\usepackage[french]{babel} % guillemets
\usepackage[T1]{fontenc}

\usepackage{amsmath}
\usepackage{amssymb}
\usepackage{float}
\usepackage{newfloat}
\usepackage{tocloft}

\newcommand{\listofdiagramsname}{List of Diagrams}
\DeclareFloatingEnvironment[fileext=loq,
    placement={!htbp},
    name=Diagrams,
    listname=\listofdiagramsname,
]{diagram}

\addto\captionsfrench{\renewcommand{\diagramname}{\textsc{Diagramme}}}
\addto\captionsfrench{\renewcommand{\listofdiagramsname}{Liste des diagrammes}}

\usepackage{svg}
\graphicspath{{figures/}}

\usepackage{tikz}
\usetikzlibrary{babel}
\usetikzlibrary{calc}
\usetikzlibrary{decorations}
\usetikzlibrary{calligraphy}%Braces
\usetikzlibrary{angles}
\usetikzlibrary{quotes}

\usepackage{tkz-euclide}

\usepackage{siunitx}
\sisetup{%
     output-decimal-marker = {,},
     inter-unit-product = \ensuremath{{}\cdot{}},
     per-mode = fraction,
}
\DeclareSIUnit\knot{\text{n\oe uds}}

\usepackage{enumerate}

\usepackage[french]{datetime2}
\DTMnewdatestyle{qcdate}{%
    \renewcommand{\DTMdisplaydate}[4]{\number##3{} \DTMenglishmonthname{##2} \number##1}%
  \renewcommand{\DTMDisplaydate}{\DTMdisplaydate}%
}
\DTMsetdatestyle{qcdate}

\usepackage[skip=\medskipamount]{parskip}

\usepackage{titlesec}
\titlelabel{\thetitle.\quad}

\newcommand{\qed}{\rlap{$\qquad \blacksquare$}}

\sloppy

\title{Exercices---Plan inclin\'e, forces et \'energie}
\date{\today}
\author{Nicolas Landucci}

\begin{document}
\maketitle

\section{D\'ecollage}
    Un avion d'une masse de \qty{7.50e4}{\kilogram} s'\'elance sur une piste de \qty{2.00}{\kilo\metre} de longueur pour d\'ecoller. Ses moteurs fournissent une pouss\'ee de \qty{2.42e5}{\newton}; le coefficient de frottement cin\'etique entre les pneus de l'appareil et la piste est de \num{0.03} (r\'esistance de roulement).

    \begin{figure}[H]
        \centering
        \resizebox{\linewidth}{!}{
            \begin{tikzpicture}
                \node[inner sep=0pt, above right] at (0,0)
                    {\includesvg[width=.125\textwidth]{a320.svg}};
                \draw (0,0) -- (8,0);
            \end{tikzpicture}
        }
        
        \caption{Un avion sur une piste de d\'ecollage.}
    \end{figure}

    \begin{enumerate}[a)]
    \item
        Quelle sera l'acc\'el\'eration de l'avion?
    \item
        S'il quitte le sol lorsqu'il atteint la vitesse de \qty{155}{\knot} (\qty{79.7}{\metre\per\second}), sur quelle distance l'avion roulera-t-il sur la piste?
        
    \begin{figure}[H]
        \centering
        \resizebox{\linewidth}{!}{
            \begin{tikzpicture}
                \node[inner sep=0pt, above right, rotate=10] at (2,0.2)
                    {\includesvg[width=.125\textwidth]{a320.svg}};
                \draw (0,0) -- (8,0);
            \end{tikzpicture}
        }
        
        \caption{Le m\^eme avion, apr\`es d\'ecollage.}
    \end{figure}
    \item
        Une fois que l'avion d\'ecolle, il monte en conservant la m\^eme vitesse (\qty{155}{\knot}, \qty{79.7}{\metre\per\second}) et la m\^eme pouss\'ee. Si la r\'esistance de l'air est n\'egligeable, quel est l'angle de l'avion?
    \item
        Lorsqu'il sera rendu \`a l'extr\'emit\'e de la piste, quelle sera la hauteur de l'avion?
    \end{enumerate}

    \clearpage
    \subsection*{Solution}

    
    \begin{enumerate}[a)]
            
    
    \item
        Quelle sera l'acc\'el\'eration de l'avion?
        \subsubsection*{Diagramme de corps libre}
        \begin{diagram}[H]
            \centering
            \resizebox{\linewidth}{!}{
                \begin{tikzpicture}
                    \node[inner sep=0pt, above right, opacity=0.25] (plane) at (1,0) 
                        {\includesvg[width=6cm]{a320.svg}};
                    \coordinate (comtop) at ($(plane.center)!0.25!(plane.east)$);
                    \coordinate (combot) at ($(plane.south)!0.25!(plane.south east)$);
                    \coordinate (com) at ($(comtop)!0.5!(combot)$);
                    \draw[draw opacity=0.25] (0,0) -- (8,0);
                    \draw[-latex, color=blue] (com) -- ++(0,1) node[near end,right] {$F_N$};
                    \draw[-latex, color=blue] (com) -- ++(0,-1) node[near end,right] {$F_g$};
                    \draw[-latex, color=red] (com) -- ++(1,0) node[near end,above] {$F$};
                    \draw[-latex, color=red] (com) -- ++(-0.5,0) node[near end,above] {$F_{F_c}$};
                    \filldraw[black] (com) circle (1.5pt);
                \end{tikzpicture}
            }
            
            \caption{Un avion sur une piste de d\'ecollage.}
        \end{diagram}
        \begin{align*}
            m &= \qty{7.50e4}{\kilogram} \\
            g &= \qty{9.8}{\metre\per\second\squared} \\
            F &= \qty{2.42e5}{\newton} \\
            \mu_c &= \num{0.03}
        \end{align*}
        
        \paragraph*{Calculer $F_N$}
        \begin{align*}
            \sum F_y &= 0 \\
            F_N - F_g &= 0 \\
            F_N &= F_g \\
            F_N &= mg \\
            F_N &= \left(\qty{7.50e4}{\kilogram}\right) \left(\qty{9.8}{\metre\per\second\squared}\right) \\
            F_N &= \qty{7.35e5}{\newton} \end{align*}
            
        \paragraph*{Calculer $a$}
        \begin{align*}
            \sum F_x &= ma \\
            F - F_{F_c} &= ma \\
            F - \mu_cF_N &= ma \\
            \qty{2.42e5}{\newton} - \num{0.03}\left(\qty{7.35e5}{\newton}\right) &= \left(\qty{7.50e4}{\kilogram}\right)a \\
            \frac{\qty{2.42e5}{\newton} - \num{0.03}\left(\qty{7.35e5}{\newton}\right)}{\qty{7.50e4}{\kilogram}} &= a \\
            \qty{2.93}{\metre\per\second\squared} &= a \qed
        \end{align*}
        
    \filbreak
    \item
        S'il quitte le sol lorsqu'il atteint la vitesse de \qty{155}{\knot} (\qty{79.7}{\metre\per\second}), sur quelle distance l'avion roulera-t-il sur la piste?
        \begin{align*}
            v_i &= \qty{0}{\metre\per\second} \\
            v_f &= \qty{79.7}{\metre\per\second} \\
            a &= \qty{2.93}{\metre\per\second\squared}
        \end{align*}

        \paragraph*{Calculer $\Delta x$}
        \begin{align*}
            v_f^2 &= v_i^2 + 2a\Delta x \\
            v_f^2 - v_i^2 &= 2a\Delta x \\
            \left(\qty{79.7}{\metre\per\second}\right)^2 - \left(\qty{0}{\metre\per\second}\right)^2 &= 2\left(\qty{2.93}{\metre\per\second\squared}\right)\Delta x \\
            \frac{\left(\qty{79.7}{\metre\per\second}\right)^2 - \left(\qty{0}{\metre\per\second}\right)^2}{2\left(\qty{2.93}{\metre\per\second\squared}\right)} &= \Delta x \\
            \qty{1.084e3}{\metre} &= \Delta x \tag{Plus de pr\'ecision pour les calculs suivants} \\
            \qty{1.08e3}{\metre} &= \Delta x \qed \tag{3 C.\,S. pour la r\'eponse}\\
        \end{align*}
        
    \filbreak
    
    \item
        Une fois que l'avion d\'ecolle, il monte en conservant la m\^eme vitesse (\qty{155}{\knot}, \qty{79.7}{\metre\per\second}) et la m\^eme pouss\'ee. Si la r\'esistance de l'air est n\'egligeable, quel est l'angle de l'avion?
        \subsubsection*{Diagramme de corps libre}
        \begin{diagram}[H]
            \centering
            \resizebox{\linewidth}{!}{
                \begin{tikzpicture}
                    \node[inner sep=0pt, above right, opacity=0.25, rotate=20] (plane) at (1,0.5) 
                        {\includesvg[width=6cm]{a320.svg}};
                    \coordinate (comtop) at ($(plane.center)!0.25!(plane.east)$);
                    \coordinate (combot) at ($(plane.south)!0.25!(plane.south east)$);
                    \coordinate (com) at ($(comtop)!0.5!(combot)$);
                    \draw[draw opacity=0.25] (0,0) -- (8,0);
                    \begin{scope}[rotate=20]
                        \draw[-latex, color=blue] (com) -- ++(0,{1.5*cos(20)}) node[near end,right] {$F_N$};
                        \draw[-latex, color=blue] (com) -- ++(0,{-1.5*cos(20)}) node[near end,right] {$F_{g_y}$};
                        \draw[-latex, color=red] (com) -- ++({1.5*sin(20)},0) node[near end,above] {$F$};
                        \draw[-latex, color=red] (com) -- ++({-1.5*sin(20)},0) node[near end,left] {$F_{g_x}$};
                        \draw[-latex] ($(com) + (1,0.75)$) -- ++(1,0) node[midway,above left] {$\vec{v}$};
                        \filldraw[black] (com) circle (1.5pt);
                    \end{scope}
                \end{tikzpicture}
            }
            
            \caption{Le m\^eme avion, apr\`es d\'ecollage.}
        \end{diagram}
    
        \begin{align*}
            m &= \qty{7.50e4}{\kilogram} \\
            g &= \qty{9.8}{\metre\per\second\squared} \\
            F &= \qty{2.42e5}{\newton} \\
            a &= \qty{0}{\metre\per\second\squared} \tag{Vitesse constante}
        \end{align*}
        
        \paragraph*{Calculer $\theta$}
        \begin{align*}
            \sum F_x &= ma \\
            F - F_{g_x} &= m\left(\qty{0}{\metre\per\second\squared}\right) \\
            F - F_{g_x} &= 0 \\
            F &= F_{g_x} \\
            F &= mg\sin \theta \\
            \qty{2.42e5}{\newton} &= \left(\qty{7.50e4}{\kilogram}\right) \left(\qty{9.8}{\metre\per\second\squared}\right) \sin \theta \\
            \frac{\qty{2.42e5}{\newton}}{\left(\qty{7.50e4}{\kilogram}\right) \left(\qty{9.8}{\metre\per\second\squared}\right)} &= \sin \theta \\
            \sin^{-1} \left(\frac{\qty{2.42e5}{\newton}}{\left(\qty{7.50e4}{\kilogram}\right) \left(\qty{9.8}{\metre\per\second\squared}\right)}\right) &= \theta \\
            \qty{19.2}{\degree} &= \theta \qed
        \end{align*}
    
    \filbreak
    \item
        Lorsqu'il sera rendu \`a l'extr\'emit\'e de la piste, \`a quelle hauteur sera l'avion?
        
        \begin{diagram}[H]
            \centering
            \resizebox{\linewidth}{!}{
                \begin{tikzpicture}[scale=1.25]
                    \draw[dashed] (0,0) -- ({1084/2000*10},0) node[midway, above] {\qty{1084}{\metre}} coordinate (o);
                    \draw[decorate, decoration={calligraphic brace,mirror}] (0,-0.1) -- ++(10, 0) node[midway, below=1pt] {\qty{2000}{\metre}};
                    \draw (o) -- (10,0) node[pos=0.85, above left] {$\left(2000-1084\right)\unit{\metre}$} coordinate (a) -- (10,1.6) node[midway, right] {$h$} coordinate (b) -- cycle;
                    \tkzMarkAngle[size=1](a,o,b)
                    \tkzLabelAngle[pos=1.2](a,o,b){$\theta$}
                    \tkzMarkRightAngle[size=0.3](o,a,b)
                \end{tikzpicture}
            }
            \caption{Hauteur \`a la fin de la piste et distance restante.}
        \end{diagram}

        \begin{align*}
            \theta &= \qty{19.2}{\degree}
        \end{align*}
        
        \paragraph*{Calculer $h$}
        \begin{align*}
            \tan \theta &= \frac{h}{\left(2000-1084\right)\unit{\metre}} \\
            \tan \qty{19.2}{\degree} &= \frac{h}{\qty{916}{\metre}} \\
            \left(\tan \qty{19.2}{\degree}\right) \left(\qty{916}{\metre}\right) &= h \\
            \qty{319}{\metre} &= h \qed
        \end{align*}
        
    \end{enumerate}

    \clearpage
    \section{Freinage d'urgence}
        En freinant au seuil (freinage maximal sans que les pneus d\'erapent), quelle distance une voiture prend-elle pour passer de \qty{100}{\kilo\metre\per\hour} \`a l'arr\^et sur une route plate? Les coefficients de friction statique et cin\'etique entre le caoutchouc et l'asphalte sont respectivement de \num{0.7} et de \num{0.5}.

    \clearpage
    \subsection*{Solution}

    \'Etant donn\'e que les pneus ne d\'erapent pas, le coefficient de friction entre les roues et le sol est de \num{0.7} $\left(\mu_s\right)$.
    
    \subsubsection*{Diagramme de corps libre}
    \begin{diagram}[H]
        \centering
        \resizebox{0.5\linewidth}{!}{
            \begin{tikzpicture}
                \draw (0,0) -- (4,0);
                \draw (1.5,0) rectangle (2.5,1);
                \coordinate (com) at (2,0.5);
                \draw[-latex, color=blue] (com) -- ++(0,1) node[near end,right] {$F_N$};
                \draw[-latex, color=blue] (com) -- ++(0,-1) node[near end,right] {$F_g$};
                \draw[-latex, color=red] (com) -- ++(-0.7,0) node[near end,left] {$F_{F_s}$};
                \draw[-latex] ($(com) + (0.8,0)$) -- ++(0.7,0) node[midway,above] {$\vec{v}$};
                \filldraw[black] (com) circle (1.5pt);
            \end{tikzpicture}
        }
        \caption{Une voiture en freinage.}
    \end{diagram}
        
        
        \begin{align*}
            g &= \qty{9.8}{\metre\per\second\squared} \\
            \mu_s &= \num{0.7} \\
            v_i &= \qty{100}{\kilo\metre\per\hour} \\
            v_i &= \qty{100}{\kilo\metre\per\hour} \cdot \frac{\qty{1}{\metre\per\second}}{\qty{3.6}{\kilo\metre\per\hour}} = \qty{27.78}{\metre\per\second} \\
            v_f &= \qty{0}{\metre\per\second}
        \end{align*}
        
        \paragraph*{Calculer $F_N$}
        \begin{align*}
            \sum F_y &= 0 \\
            F_N - F_g &= 0 \\
            F_N &= F_g \\
            F_N &= mg \\
            F_N &= m \left(\qty{9.8}{\metre\per\second\squared}\right)
        \end{align*}
        
        \paragraph*{Calculer $a$}
        \begin{align*}
            \sum F_x &= ma \\
            - F_{F_s} &= ma \\
            - \mu_sF_N &= ma \\
            - \num{0.7} \cdot m \left(\qty{9.8}{\metre\per\second\squared}\right) &= ma \\
            - \num{0.7} \left(\qty{9.8}{\metre\per\second\squared}\right) &= a \tag{Simplification de $m$ des deux c\^ot\'es} \\
            \qty{-6.86}{\metre\per\second\squared} &= a
        \end{align*}
        
        \paragraph*{Calculer $\Delta x$}
        \begin{align*}
            v_f^2 &= v_i^2 + 2a\Delta x \\
            v_f^2 - v_i^2 &= 2a\Delta x \\
            \left(\qty{0}{\metre\per\second}\right)^2 - \left(\qty{27.78}{\metre\per\second}\right)^2 &= 2\left(\qty{-6.86}{\metre\per\second\squared}\right)\Delta x \\
            \frac{\left(\qty{0}{\metre\per\second}\right)^2 - \left(\qty{27.78}{\metre\per\second}\right)^2}{2\left(\qty{-6.86}{\metre\per\second\squared}\right)} &= \Delta x \\
            \qty{56.2}{\metre} &= \Delta x \qed \tag{3 C.\,S.}\\
        \end{align*}

\end{document}
